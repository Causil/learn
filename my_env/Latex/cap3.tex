\part{Matem\'aticas}

\chapter{Ecuaciones diferenciales}
En est\'e cap\'itulo vamos a entender que es una ecuaci\'on diferencial. 

Para comenzar a formular lo que es una ecuaci\'on diferencial, comencemos explorando lo que es la en s\'i, la palabra ecuaci\'on, de donde proviene, ?`C\'omo hemos trascendido los seres humanos en el pensamiento de las ecuaciones? ?`Para que sirven las ecuaciones y por ende donde aplicarlas? Es un tarea muy dura, pero lo vamos a lograr, formulando  preguntas para ir llegando a respuestas y de estas respuestas dar el salto a lo que yo llamo familiarizaci\'on intelectual. 

?`Por qu\'e planteamos ecuaciones? tratemos de acercarnos a esta pregunta observando la siguiente ecuaci\'on
\begin{equation}\label{equ1.1}
	y+5 = 0
\end{equation}

?`Qu\'e elementos involucrados conocemos en la ecuaci\'on \ref{equ1.1}?

Los n\'umeros 5 y 0, son dos enteros y la variable $y$ representando lo desconocido en la ecuaci\'on, es decir, lo que queremos encontrar, si nos enfocamos en ese sentido, llegaremos a que $y=-5$. 
\begin{equation}
	y(x)=y^{\prime}(x)
\end{equation}
Note que tenemos una igualdad de funciones, una ecuaci\'on que se cumple solo para las funciones que son soluciones de la misma. ?`Cuales son estas funciones? las funciones $c\exp(x)$, de esta manera podemos definir este tipo de ecuaciones, ?`Por qu\'e es importante las ecuaciones diferenciales? ?`Qu\'e fen\'omenos describe el c\'alculo? en intervengan, la distancia, velocidades y aceleraciones.

\begin{definicion}[Ecuaci\'on diferencial]
Una ecuaci\'on diferencial es cualquier ecuaci\'on que contiene las derivadas de una o m\'as variables dependientes con respecto a una o m\'as variables independientes.
\end{definicion}
\begin{ejemplo}
	\begin{align}
		\frac{dy}{dx} + \frac{dz}{dx} &= 2y + z, \\
		\frac{\partial^{2}u}{\partial{x^{2}}} &= \frac{\partial^{2}u}{\partial t^{2}} - 2\frac{\partial u}{\partial t} 
	\end{align}
\end{ejemplo}


\part{Ingles}

\chapter{Frases en ingl\'es}

\begin{description}
	\item[After you: ] Despu\'es de usted.
	\item[Check it out: ] \'Echale un vistazo.
	\item[I guess so: ] Supongo que si.
	\item[What a mess: ] Qu\'e desastre.
	\item[Good for you: ] Bien por ti.			
	
	to get one is hopes up 
\end{description}

\chapter{ Unknown words }
\begin{description}
	\item[Which: ]  cual ; pronombre: que.
	\item[Shall: ]  debera.
	\item [use:  ]  usar.
	\item[brief: ]  breve.
	\item[either:]	conjuci\'on : o, adjetive: Cualquiera de los dos, uno u otro; Adverbio: tambi\'en.
	\item[Roughly speaking: ] mas o menos
	\item[behave: ] comportarse
	\item[obey: ] obedecer
	\item[listed: ] listado
	\item[in that: ] en eso
	\item[for most:] para la mayor\'ia
	\item[may :] poder, ser posible.
	\item[well :] bien
	\item[ to allow for this :] para permitir esto
	\item[rather than  :] m\'as bien que
	\item[itself :] s\'i mismo
	\item[this means :] esto significa
	\item[so are :] tambi\'en lo son
	\item[such :] tal
	\item[below :] abajo
	\item[performance :] actuaci\'on, el rendimiento, desempe\~no
	\item[us out of :]	nosotros fuera de							
	\item[given :]	dado								
	\item[on this	 :]	en este							
	\item[unless :] a no ser que			
	\item[must  :] deber
	\item[we leave it :] lo dejamos
	\item[want  :] desear
	\item[dwell :] residir
	\item[happen :] suceder
	\item[least :] el menos
	\item[Often :] Con frecuencia
	\item[exposure :] exposici\'on
	\item[better :] mejor
	\item[worry :] preocuparse
	\item[about :] sobre						
	\item[vowels :] 
	\item[consonants :]

\end{description}

\chapter{Ingl\'es A1}

Marco Com\'un Europeo de Referencia para las Lenguas (MCER): 

\section{An\'alisis de habilidades}

A continuaci\'on, encontramos un an\'alisis de los diferentes componentes del idioma y las habilidades individuales que un estudiante deber\'ia tener para recibir un certificado A1 de ingles: 

{\bf Compresi\'on auditiva}

Los estudiantes de nivel A1 deberi\'an ser capaces de entender un ingl\'es est\'andar simple, siempre y cuando sea hablado con claridad por alguien paciente y dispuesto ayudar. Especificamente, deberi\'an poder reconocer frases y palabras comunes, relacionadas con ellos mismos, su entorno y aquellos cercanos a ellos. Adem\'as, deber\'ian entender cosas como los n\'umeros, direcciones y otras instrucciones muy b\'asicas en ingl\'es.

Welcome everyone to the basic English course. A one for beginners. I'm your teacher, kira Sage, nice to meet you. Who am i? I am from the U.S.A I have taught C level executives and software engineers at companies like google skhynix, Tesla, Sony and more. I have nine years of teaching experience in countries like Usa and Japan. Here are a couple of interesting facts about me. I have visited four of the six Disneyland's around the world. This is me at Star Wars, Galaxy edge in Anaheim California. Also i have Master's degree in teaching T. Saw that means I can teach teachers and learners like you in this course you will learn the alphabet ah and un sentences with it's plural forms, sentences with there are numbers, colors, subject pronouns, professions, greetings, negative and interrogative statements, possessive adjectives. Days of the week, your hobbies, questions with what's your and to review everything from this course we'll have to wrap up class. Here's you'll learn all of the course concepts through worksheets, find the worsheets in the resources section through interactive quizzes and through interactive explanations on an imaginary website called plattso plattso is where we practice english so we can make friends online. Are you ready for this  fun adventure? I'll see you in the next class.


Bienvenido todos al curso the Ingles b\'asico. A1 para comenzar 



\begin{verbatim}
Intento 1: 
Welcome everyone to the Black Sea English course A1 for beginners I
 am here t-shirt carousel nice to me you Hoang I who on my I'm from
 the USA I have to see level executive a software engineer and companies
 like Google sky and Tesla Sony and more I cannae jeers of teaching experience in countries like USA and Japan he are Kobe of interesting
 facts about me I had visits of the seeds Disneyland around the world this is me at Star Wars Galaxy Edge in Anaheim California also I have
 massive degree in teaching t so that means can teach t-shirts in Learners like you in this course you will learn the alphabet I'm on
 sentence with it's pure Air Force sentence with there are numbers caller surgeon pronounce profession where teens interrogative
 status. Osiria Jetties days of the weave your hobbies question with what you ain't to review everything from this course we wait
 we all have to grab a glass of the course Concepts world war ships find a wall sheets in the results section throw in charity
 crisis and throw into righteous play Nations oh imagine a website called black English so we can make friends only are you
 ready for this funeral bencher Adventure I also see you in the next class.
         
         
 Segundo intento: 
 welcome everyone to the basic English course. a wife for beginners at
  Food dock. Avon for beginners sorry please thought a blunt for 
  beginners. A1 for beginners. I am your teacher call Ma Kira say Asahi 
  call Mama, nice to meet you. who I who am I. I am from the USA. I 
  hopped out I have towels. IHOP takeout see Liberty Security in Sulphur 
  and genius and companies like Google Skynet call mistakes Le gamma 
  Sony a moor Anne Moore Anne Moore Sony and more. I have 9 years of 
  teaching experience in countries like USA in Japan. Cher here. T R A 
  Copley. here are a couple off interesting facts about me. I had visit 
  visit. I had visited full of the sea Disneyland around the world. this 
  is me and it's sad worse, Galaxy Edge in Anaheim California. also I 
  have. also I had Mercer metzer Meister Meister mustard master. also I 
  heart muscle is degree and teach t. all soup IHOP Massillon he's they 
  greet and teach tea. soda means I can teach t-shirts in layers
\end{verbatim} 

\url{https://www.youtube.com/watch?v=EVNhOAEW784}

\url{https://howjsay.com/how-to-pronounce-taught}
Para pronuciar Taught comience colocando su lengua detr\'as de sus dientes superiores y agregue el sonidoo AH corto y abierto termine con una T colocando su lengua detr\'as de sus dientes superiores



Numbers class Two: 

Welcome to the 1st Module. This english fundamentals in this module.
In this Module you will learn the 
\begin{enumerate}
	\item alphabet.
	\item A and N 
	\item sentences with it's plural forms
	\item  and sentences with there are
\end{enumerate}
 this is the alphabet. Now let's go and learn on platzo.
 
 Welcome to plattzo. Now loading oh do you know what this is?
 
 it's capture. We need letters to unlock the capture. Well let's learn in this class. 
 
 You will learn the letters of the english alphabet. In the English Alphabet there are 26 letters. Let's practice. I will say a letter an repeat after me. Ready, let's go. A B Bravo. C D E. Excellent F G. Googd job H I J K, keep it up. L M magnificent N. Nice work. O P perfect Q R S super T terrific. U V very goog W wonderful X. Why Z in some countries this letter is zed.
 
 Now, you know the letters of the alphabet? A B C D E F G H I J K L M N O P Q R S T u v w x, y z. OR ZED. 	Let's use these letters to unlock the capture on plattzo.
 
 The capture says a phrase, not a robot. Let's spell the phrase letter by letter N O T A R O B O T. You did it. You unlock the capture. 
 
 Great job. it's your turn. Here is the alphabet but some letters are missing. 
 
 Type, the missing letters in the discussion panel. 
 Don't forget to download your worksheet from the resources section that worksheet is your plattzo profile. Complete the user name area of your plattzo profile on your woorksheet, keep this worksheet with you at all times. We will complete your plattzo profile in each module of this course. So keep ir ready for extra practice. Type your user name in the discussion panel.
 
 My user name is Kira K I R A. I'll see you in the next class.
 
 
 Email: jcausilmartinez@gmail.com
 
 Number phone: 314 796 81 19.
 
 User name: Javier Andres Causil Martínez


My introducction in is Spanish:  

Hola mi nombre es Javier Andr\'es Causil Mart\'inez, soy de Colombia del departamento de Antioquia en la ciudad de Caucasia. Tengo 27 a\~nos, soy Matem\'atico  de la Universidad de Antioquia y programador, me gusta la m\'usica, hacer deporte, aprender cosas nuevas. Siento gran pasi\'on por la m\'usica cl\'asica. 



\subsection{A and AN}

A and AN refer to only one person, place, or thing, or noun.

An is an article that we use with vowels. Vowels are ``a'', ``e'', ``i'', ``o'', ``u''

Use ``an'' with these vowels. Now, let's them together on Platzo.

``A'' is the article we use with consonants.

Be careful! there is an exception. The most common exception is with the letter U.

Now you can use ``a'' and ``an'' with vowels and consonants.

\subsection{IT'S SENTENCES}

In this class, you will learn sentences with ``it's''.

First, use ``it's'' or ``'it is' with a noun, 

it is + a person: a thing: an item: a place

We can make sentences with ``it's, an article, and a noun''.

In this class, you wil learn forms plurals, 

\subsection{Plural forms}

The most common plural form is to add ``s'' at the end of the word. Para otras palabras que terminan con ``S,SH,CH,X'' y ``Z'', we add ``es'' to the end of those words.

Words with ``y'' are special. because when a ``y'' is next to a vowel, we just add ``S'', when a ``y'' is next to a consonant, we change the ``y'' to ``i'' and add ``es''.

\part{Lecturas de libros}

\section{Ultralearning}

Comienza con la ilusi\'on de estudiar en el MIT, pero t\'ermina estudiando negocios en la Universidad de Manitoba, una escuela Canadiense de rango medio, ya que era la que pod\'ia pagar. Pero al terminar la carrera concluyo que se equivoco de carrera cuatro a\~nos mas tarde se da cuenta de una especializaci\'on en administraci\'on y de inform\'atica 

\subsection{Why Ultralearning Matters}

Cu\'ando aplicar el ultraaprendizaje 

\begin{enumerate}
	\item es seguir ultralearning a tiempo parcial.
	\item es buscar el ultraaprendizaje durante las brechas en el trabajo y la escuela.
	\item Es integrar los principios del ultraaprendizaje en el tiempo y la energía que ya dedica al aprendizaje.
	
\end{enumerate}


\subsection{El valor del ultraaprendizaje}

La capacidad de adquirir habilidades duras de manera efectiva y eficiente es inmensamente valiosa. No solo eso, sino que las tendencias actuales en econom\'ia, educaci\'on y tecnolog\'ia van a exacerbar la diferencia entre quienes tienen esta habilidad y quienes no la tienen.

\subsection{Como convertirse en un ultraaprendiz}

Primeros pasos de un ultraaprendiz novato


Principios para convertirse en ultraaprendiz



\begin{verbatim}
	1. METALEARNING: PRIMERO DIBUJAR UN MAPA. Comience por aprender cómo aprender el tema o la habilidad que desea abordar. Descubra cómo hacer una buena investigación y cómo aprovechar sus competencias pasadas para aprender nuevas habilidades más fácilmente.
	
	2. ENFOQUE: AFILA TU CUCHILLO. Cultivar la capacidad de concentración. Dedique períodos de tiempo en los que pueda concentrarse en el aprendizaje y haga que sea fácil hacerlo.
	
	3. SERIEDAD: VAYA DERECHO. Aprende haciendo aquello en lo que quieres ser bueno. No lo cambie por otras tareas, solo porque son más convenientes o cómodas.
	
	4. EJERCICIO: ATACA TU PUNTO MÁS DÉBIL. Sea implacable en la mejora de sus puntos más débiles. Divida las habilidades complejas en partes pequeñas; luego domine esas partes y vuelva a construirlas juntas.
	
	5. RECUPERACIÓN: PRUEBA PARA APRENDER. Las pruebas no son simplemente una forma de evaluar el conocimiento, sino una forma de crearlo. Ponte a prueba antes de sentirte seguro y oblígate a recordar activamente la información en lugar de revisarla pasivamente.
	
	6.COMENTARIOS: NO EVITE LOS GOLPES. La retroalimentación es dura e incómoda. Sepa cómo usarlo sin dejar que su ego se interponga en el camino. Extraiga la señal del ruido, para que sepa a qué prestar atención y qué ignorar.
	
	7. RETENCIÓN: NO LLENE UN CUBO CON FUGAS. Entiende lo que olvidas y por qué. Aprende a recordar las cosas no solo por ahora sino para siempre.
	
	8. INTUICIÓN: PROFUNDICE ANTES DE CONSTRUIR. Desarrolla tu intuición a través del juego y la exploración de conceptos y habilidades. Entiende cómo funciona el entendimiento, y no recurras a trucos baratos de memorización para evitar saber las cosas en profundidad.
	
	9. EXPERIMENTACIÓN: EXPLORA FUERA DE TU ZONA DE CONFORT. Todos estos principios son solo puntos de partida. El verdadero dominio proviene no solo de seguir el camino recorrido por otros, sino de explorar posibilidades que aún no han imaginado.
\end{verbatim}

\section{Principio 1}

\subsection{What Is Metalearning?}

Aprehender sobre el aprendizaje

El poder de su mapa de metaaprendizaje

Ser capaz de ver c\'omo funciona un tema, qu\'e tipo de habilidades e informaci\'on se deben dominar y qu\'e m\'etodos est\'an disponibles para hacerlo de manera m\'as efectiva es la base del \'exito de todos los proyectos de ultraaprendizaje.

\begin{verbatim}
	Una vez que haya entendido por qué está aprendiendo, puede comenzar a ver cómo se estructura el conocimiento en su materia. Una buena manera de hacer esto es escribir en una hoja de papel tres columnas con los títulos “Conceptos”, “Hechos” y “Procedimientos”. Luego haga una lluvia de ideas sobre todas las cosas que necesitará aprender. No importa si la lista está perfectamente completa o precisa en esta etapa. Siempre puedes revisarlo más tarde. Su objetivo aquí es obtener un primer pase difícil. Una vez que comience a aprender, puede ajustar la lista si descubre que sus categorías no son del todo correctas.
\end{verbatim}

Primera columna de conceptos  si algo necesita ser entendido, no solo memorizado, lo pongo en esta columna en lugar de la segunda columna de hechos.
\begin{verbatim}
	1. METALEARNING: PRIMERO DIBUJAR UN MAPA. Comience por aprender cómo aprender el tema o la habilidad que desea abordar. Descubra cómo hacer una buena investigación y cómo aprovechar sus competencias pasadas para aprender nuevas habilidades más fácilmente.
	
	2. ENFOQUE: AFILA TU CUCHILLO. Cultivar la capacidad de concentración. Dedique períodos de tiempo en los que pueda concentrarse en el aprendizaje y haga que sea fácil hacerlo. 
	
	3. SERIEDAD: VAYA DERECHO. Aprende haciendo aquello en lo que quieres ser bueno. No lo cambie por otras tareas, solo porque son más convenientes o cómodas.
	
	4. EJERCICIO: ATACA TU PUNTO MÁS DÉBIL. Sea implacable en la mejora de sus puntos más débiles. Divida las habilidades complejas en partes pequeñas; luego domine esas partes y vuelva a construirlas juntas.
	
	5. RECUPERACIÓN: PRUEBA PARA APRENDER. Las pruebas no son simplemente una forma de evaluar el conocimiento, sino una forma de crearlo. Ponte a prueba antes de sentirte seguro y oblígate a recordar activamente la información en lugar de revisarla pasivamente. 
	
	6.COMENTARIOS: NO EVITE LOS GOLPES. La retroalimentación es dura e incómoda. Sepa cómo usarlo sin dejar que su ego se interponga en el camino. Extraiga la señal del ruido, para que sepa a qué prestar atención y qué ignorar.
	
	7. RETENCIÓN: NO LLENE UN CUBO CON FUGAS. Entiende lo que olvidas y por qué. Aprende a recordar las cosas no solo por ahora sino para siempre. 
	
	8. INTUICIÓN: PROFUNDICE ANTES DE CONSTRUIR. Desarrolla tu intuición a través del juego y la exploración de conceptos y habilidades. Entiende cómo funciona el entendimiento, y no recurras a trucos baratos de memorización para evitar saber las cosas en profundidad. 
	
	9. EXPERIMENTACIÓN: EXPLORA FUERA DE TU ZONA DE CONFORT. Todos estos principios son solo puntos de partida. El verdadero dominio proviene no solo de seguir el camino recorrido por otros, sino de explorar posibilidades que aún no han imaginado. 
\end{verbatim}

\section{Principio 1}

\subsection{What Is Metalearning?}

Aprehender sobre el aprendizaje

El poder de su mapa de metaaprendizaje

Ser capaz de ver c\'omo funciona un tema, qu\'e tipo de habilidades e informaci\'on se deben dominar y qu\'e m\'etodos est\'an disponibles para hacerlo de manera m\'as efectiva es la base del \'exito de todos los proyectos de ultraaprendizaje.

\begin{verbatim}
	Una vez que haya entendido por qué está aprendiendo, puede comenzar a ver cómo se estructura el conocimiento en su materia. Una buena manera de hacer esto es escribir en una hoja de papel tres columnas con los títulos “Conceptos”, “Hechos” y “Procedimientos”. Luego haga una lluvia de ideas sobre todas las cosas que necesitará aprender. No importa si la lista está perfectamente completa o precisa en esta etapa. Siempre puedes revisarlo más tarde. Su objetivo aquí es obtener un primer pase difícil. Una vez que comience a aprender, puede ajustar la lista si descubre que sus categorías no son del todo correctas.
	
	Primera columna de conceptos  si algo necesita ser entendido, no solo memorizado, lo pongo en esta columna en lugar de la segunda columna de hechos.
	
	En la segunda columna, escribe todo lo que necesites memorizar. Los hechos son cualquier cosa que sea suficiente si puedes recordarlos. No es necesario que los entienda demasiado profundamente, siempre y cuando pueda recordarlos en las situaciones adecuadas. 
	
	En la tercera columna, escribe todo lo que necesites practicar. Los procedimientos son acciones que deben realizarse y es posible que no impliquen mucho pensamiento consciente.
	
	Una vez que haya terminado su lluvia de ideas, subraye los conceptos, hechos y procedimientos que serán más desafiantes. Esto le dará una buena idea de cuáles serán los principales cuellos de botella de aprendizaje y puede comenzar a buscar métodos y recursos para superar esas dificultades.
	
	Ahora que ha respondido dos preguntas, por qué está aprendiendo y qué está aprendiendo, es hora de responder la pregunta final: ¿Cómo va a aprenderlo? Sugiero seguir dos métodos para responder cómo aprenderá algo: Benchmarking y el Método de énfasis/exclusión. (Benchmarking and the Emphasize/Exclude Method.)
	
	Las luchas con el enfoque que tienen las personas generalmente vienen en tres variedades amplias: comenzar, mantener y optimizar la calidad del enfoque de uno. Los ultraalumnos son implacables a la hora de encontrar soluciones para manejar estos tres problemas, que forman la base de la capacidad de concentrarse bien y aprender profundamente.
	
	Examinemos algunas de las tácticas que usan los ultraestudiantes para maximizar este principio y aprovechar las deficiencias de la educación más típica. 
	
	Táctica 1: aprendizaje basado en proyectos Muchos ultraestudiantes optan por proyectos en lugar de clases para aprender las habilidades que necesitan.
	
	Táctica 2: Aprendizaje inmersivo
	
	La inmersión es el proceso de rodearse del entorno objetivo en el que se practica la habilidad. Esto tiene la ventaja de requerir una cantidad de práctica mucho mayor de lo que sería típico, además de exponerlo a una gama más completa de situaciones en las que se aplica la habilidad.
	
	Táctica 3: El método del simulador de vuelo
	
	Es simular el entorno, aproximarse al entorno real a través de uno muy parecido.
	
	Táctica 4: El enfoque exagerado
	
	El último método que encontré para mejorar la franqueza es aumentar el desafío, de modo que el nivel de habilidad requerido esté completamente contenido dentro de la meta establecida. 
	
	Aprenda directamente de la fuente
	
	El aprendizaje directo es uno de los sellos distintivos de muchos de los proyectos exitosos de ultraaprendizaje que he encontrado, particularmente por lo diferente que puede ser del estilo de educación al que la mayoría de nosotros estamos acostumbrados.
	
	Los ultraestudiantes, por el contrario, emplean con frecuencia lo que llamaré el enfoque Direct-Then-Drill.
	
	El primer paso es tratar de practicar la habilidad directamente. Esto significa averiguar dónde y cómo se usará la habilidad y luego tratar de igualar esa situación lo más cerca posible al practicar. 
	
	El siguiente paso es analizar la habilidad directa y tratar de aislar los componentes que son pasos determinantes en su rendimiento o subhabilidades que encuentra difíciles de mejorar porque hay demasiadas otras cosas sucediendo para que usted se concentre en ellas.
	
	El paso final es volver a la práctica directa e integrar lo que has aprendido. Esto tiene dos propósitos. La primera es que, incluso en ejercicios bien diseñados, habrá problemas de transferencia debido al hecho de que lo que antes era una habilidad aislada debe trasladarse a un contexto nuevo y más complejo. Piense en esto como si estuviera construyendo el tejido conectivo para unir los músculos que fortaleció por separado. La segunda función de este paso es verificar si su ejercicio fue bien diseñado y apropiado. Muchos intentos de aislar un ejercicio pueden terminar en fracaso porque el ejercicio realmente no llega al corazón de lo que era difícil en la práctica real. 
	
	Tácticas para diñar simulacros: 
	
	Tu primer proyecto de ultraaprendizaje
	
	El comienzo es siempre hoy.
	-Mary Shelley
	
	A estas alturas, probablemente esté ansioso por comenzar su propio proyecto de ultraaprendizaje. ¿Qué cosas podría aprender que ha pospuesto debido a temores de insuficiencia, frustración o falta de tiempo? ¿Qué viejas habilidades podrías llevar a nuevas alturas? El mayor obstáculo para el ultraaprendizaje es simplemente que la mayoría de las personas no se preocupan lo suficiente por su propia autoeducación como para comenzar. Como has leído hasta aquí, dudo que eso sea cierto para ti. El aprendizaje, en cualquier forma que adopte, es algo importante para usted. La pregunta es si esa chispa de interés se encenderá en una llama o se apagará prematuramente. Los proyectos de ultraaprendizaje no son fáciles. Requieren planificación, tiempo y esfuerzo. Sin embargo, las recompensas valen el esfuerzo. Ser capaz de aprender cosas difíciles de manera rápida y efectiva es una habilidad poderosa. Un proyecto exitoso tiende a conducir a otros. Por lo general, es el primer proyecto que requiere más reflexión y cuidado. Un plan sólido, bien investigado y bien ejecutado puede brindarle la confianza para enfrentar desafíos más difíciles en el futuro. Un intento fallido no es un desastre, pero puede volverlo reacio a emprender proyectos futuros de naturaleza similar. En este capítulo, me gustaría contarles todo lo que he aprendido sobre cómo hacerlo bien.
	
	Paso 1: Haga su investigación
	
	El primer paso en cualquier proyecto es hacer la investigación de metalaprendizaje requerida para darle un buen punto de partida. Planificar con anticipación evitará muchos problemas y evitará que tengas que hacer cambios drásticos en tu plan de aprendizaje antes de que hayas comenzado a progresar. La investigación es un poco como empacar una maleta para un viaje largo. Es posible que no traiga los artículos correctos o que olvide algo y necesite comprarlo en el camino. Sin embargo, pensar con anticipación y empacar sus maletas correctamente evitará muchos errores más adelante. Su lista de verificación de "empaque" de ultralearning debe incluir, como mínimo:
	
	1. QUÉ TEMA VAS A APRENDER Y SU ALCANCE APROXIMADO.
	
	Obviamente, ningún proyecto de aprendizaje puede comenzar a menos que descubras lo que quieres aprender. En algunos casos, esto es obvio. En otros, es posible que deba investigar más para identificar qué habilidad o conocimiento sería más valioso. Si su objetivo es aprender algo instrumentalmente (para iniciar un negocio, obtener una promoción, investigar para un artículo), aprender lo que necesita aprender es importante y le sugerirá qué tan amplio y profundo debe llegar. Sugiero comenzar con un alcance más bien estrecho, que puede expandirse a medida que avanza. “Aprender suficiente chino mandarín para mantener una conversación de quince minutos sobre temas sencillos” es mucho más limitado que “Aprender chino”, que puede incluir leer, escribir, estudiar historia y más.
	
	2. LOS PRINCIPALES RECURSOS QUE VAS A UTILIZAR.
	
	Esto incluye libros, videos, clases, tutoriales, guías e incluso personas que servirán como mentores, entrenadores y compañeros. Aquí es donde decides cuál será tu punto de partida. Ejemplos: “Voy a leer y completar los ejercicios de un libro de programación en Python para principiantes” o “Voy a aprender español a través de tutorías online a través de italki.com” o “Voy a practicar el dibujo haciendo bocetos.” En algunas asignaturas, los materiales estáticos determinarán cómo proceder. En otros, serán apoyos para respaldar su práctica. En cualquier caso, deben identificarse, comprarse, prestarse o inscribirse antes de comenzar.
	
	
	3. UN PUNTO DE REFERENCIA DE CÓMO OTROS HAN APRENDIDO CON ÉXITO ESTA HABILIDAD O TEMA.
	
	Casi cualquier habilidad popular tiene foros en línea donde aquellos que han aprendido la habilidad previamente pueden compartir sus enfoques. Debe identificar las cosas que otras personas que han aprendido la habilidad han hecho para aprenderla. Esto no significa que debas seguir exactamente sus pasos, pero evitará que te pierdas algo importante por completo. El método de entrevista con expertos del capítulo 4 proporciona un buen método para hacer un seguimiento de esto.
	
	
	Paso 2: programe su tiempo
	
	Su proyecto de ultraaprendizaje no necesita ser un esfuerzo intensivo de tiempo completo para tener éxito. Sin embargo, requerirá una inversión de tiempo, y es mejor decidir cuánto tiempo está dispuesto a dedicar al aprendizaje por adelantado que simplemente esperar que encontrará el tiempo más tarde. Hay dos buenas razones para planificar su agenda con anticipación. La primera es que de esta manera subconscientemente priorizas tu proyecto al establecerlo en tu calendario por delante de otras cosas. La segunda es que el aprendizaje suele ser frustrante y casi siempre es más fácil hacer clic en Facebook, Twitter o Netflix. Si no dedica tiempo a aprender, será mucho más difícil reunir la motivación para hacerlo. La primera decisión que debes tomar es cuánto tiempo vas a dedicar. Esto es a menudo dictado por su horario. Es posible que tenga una brecha en el empleo que le permita un aprendizaje intensivo, pero solo por un mes. Alternativamente, puede tener un horario completo que le permita dedicar solo unas pocas horas por semana a aprender algo nuevo. Cualquiera que sea el tiempo que pueda comprometer, decídalo con anticipación. La segunda decisión que debes tomar es cuándo vas a aprender. ¿Durante unas horas el domingo? ¿Despertando una hora antes y poniendo el tiempo antes del trabajo? ¿Por la tarde? ¿Durante las pausas para el almuerzo? Una vez más, lo mejor es hacer lo que sea más fácil según tu horario. Recomiendo establecer un horario constante que sea el mismo todas las semanas, en lugar de tratar de adaptarse al aprendizaje cuando pueda. La consistencia genera buenos hábitos, reduciendo el esfuerzo requerido para estudiar. Si no tiene absolutamente ninguna opción, un cronograma ad hoc es mejor que ninguno, pero requerirá más disciplina para mantenerlo. Si tiene cierta flexibilidad en su horario, es posible que desee optimizarlo. Los fragmentos de tiempo espaciados más cortos son mejores para la memoria que los fragmentos abarrotados. Sin embargo, algunos tipos de tareas, como la escritura y la programación, tienen un tiempo de calentamiento prolongado que puede beneficiarse de períodos de tiempo ininterrumpidos más prolongados. La mejor manera de descubrir qué es lo mejor para ti es practicar; si encuentra que toma mucho tiempo calentarse, opte por espacios más largos en su horario. Si descubre que puede comenzar a trabajar a los pocos minutos de comenzar, los períodos de tiempo más cortos serán útiles para la retención a largo plazo. La tercera decisión que debe tomar es la duración de su proyecto. Por lo general, prefiero los compromisos más cortos a los más largos porque son más fáciles de cumplir. Un proyecto intensivo que dura un mes tiene menos interrupciones potenciales de la vida o de su motivación cambiando y menguando. Si tiene un gran objetivo que desea lograr que no se puede lograr en un período de tiempo corto, le sugiero que lo divida en varios más pequeños de unos pocos meses cada uno. Finalmente, tome toda esta información y póngala en su calendario. Programar con anticipación todas las horas de trabajo del proyecto tiene importantes beneficios logísticos y psicológicos. Logísticamente, esto lo ayudará a detectar posibles conflictos en su horario debido a vacaciones, trabajo o eventos familiares. Psicológicamente, le ayudará a recordar y actuar de acuerdo con su plan inicial mejor que si estuviera escrito en una hoja de papel metida en un cajón del escritorio. Además, el acto de programar demuestra tu seriedad al hacer el proyecto. Puedo recordar claramente que escribí mi horario de estudio hipotético antes de comenzar el MIT Challenge. Me tenía despierto y estudiando a las 7 a. m. y trabajando hasta las 6 p. m., con solo un breve descanso para almorzar. Aunque mi horario real, en la práctica, rara vez alcanzaba ese ideal (incluso en mis primeros días más intensivos, casi nunca llegaba a las once horas seguidas), el mero hecho de escribir el horario me ayudó a prepararme psicológicamente para el proyecto que tenía por delante. Si no está dispuesto a dedicar tiempo a su calendario, es casi seguro que no está dispuesto a dedicar tiempo a estudiar. Si estás dudando en esta etapa, es una buena señal de que tu corazón no está realmente en el lugar correcto para comenzar. Como paso extra, para aquellos que se embarcan en proyectos más largos de seis meses o más, les recomiendo hacer una semana piloto de su agenda. Esto es simple: pruebe su horario durante una semana antes de comprometerse con él. Esto le dará conocimiento de primera mano de lo difícil que será y evitará el exceso de confianza. Si ya se siente agotado después de la primera semana, es posible que deba hacer ajustes. No hay vergüenza en volver atrás y reorganizar su plan para que se ajuste mejor a su vida. Hacer este tipo de ajuste es mucho mejor que darse por vencido a mitad de camino porque su plan estaba condenado al fracaso desde el principio.
	
	Paso 3: Ejecute su plan
	
	Sea cual sea el plan con el que comenzó, ahora es el momento de hacerlo. Ningún plan es perfecto y puede darse cuenta de que lo que está haciendo para aprender se aparta del ideal, según lo establecido por los principios de ultraaprendizaje. Puede notar que su plan se basa demasiado en la lectura pasiva en lugar de la práctica de recuperación. Puede ver que la forma en que está practicando es un desvío sinuoso lejos de donde realmente querrá usarlo. Puede sentir como si estuviera olvidando cosas o memorizándolas sin entenderlas realmente. Esta bien. En algunos casos, no podrá tener el enfoque de aprendizaje perfecto porque no existen los recursos para hacerlo. Sin embargo, volverse sensible a cómo la forma en que está aprendiendo no está alineada con los principios es una buena manera de sentir los cambios que puede hacer para mejorarlo. Aquí hay algunas preguntas que debe hacerse para determinar si se está saliendo del camino. ideal:
	
	1. METALAPRENDIZAJE.
	¿He investigado cuáles son las formas típicas de aprender este tema o habilidad? ¿He entrevistado a estudiantes exitosos para ver qué recursos y consejos pueden recomendar? ¿He dedicado alrededor del 10 por ciento del tiempo total a preparar mi proyecto?
	
	2. ENFOQUE. ¿Estoy concentrado cuando paso tiempo aprendiendo, o estoy haciendo múltiples tareas y distraído? ¿Me estoy saltando las sesiones de aprendizaje o procrastinando? Cuando comienzo una sesión, ¿cuánto tiempo pasa antes de que esté en un buen flujo? ¿Cuánto tiempo puedo mantener ese enfoque antes de que mi mente comience a divagar? ¿Qué tan aguda es mi atención? ¿Debería ser más concentrado para la intensidad o más difuso para la creatividad?
	
	3. SERIEDAD. ¿Estoy aprendiendo la habilidad de la forma en que eventualmente la usaré? Si no, ¿qué procesos mentales faltan en mi práctica que existen en el entorno real? ¿Cómo puedo practicar la transferencia del conocimiento que aprendo de mi libro/clase/video a la vida real?
	
	4. TALADRO. ¿Paso tiempo centrándome en los puntos más débiles de mi desempeño? ¿Cuál es el paso limitante de la velocidad que me está frenando? ¿Siento que mi aprendizaje se está desacelerando y que hay demasiados componentes de la habilidad para dominar? Si es así, ¿cómo puedo dividir una habilidad compleja para trabajar en componentes más pequeños y manejables?
	
	5. RECUPERACIÓN. ¿Paso la mayor parte de mi tiempo leyendo y revisando, o estoy resolviendo problemas y recordando cosas de memoria sin mirar mis notas? ¿Tengo alguna forma de probarme a mí mismo o simplemente asumo que lo recordaré? ¿Puedo explicar con éxito lo que aprendí ayer, la semana pasada, hace un año? ¿Cómo sé si puedo?
	
	6. RETROALIMENTACIÓN. ¿Estoy recibiendo comentarios honestos sobre mi desempeño desde el principio, o estoy tratando de esquivar los golpes y evitar las críticas? ¿Sé bien lo que estoy aprendiendo y lo que no? ¿Estoy usando la retroalimentación correctamente o estoy reaccionando de forma exagerada a los datos ruidosos? 7. RETENCIÓN. ¿Tengo un plan para recordar lo que estoy aprendiendo a largo plazo? ¿Estoy espaciando mi exposición a la información para que se mantenga por más tiempo? ¿Estoy convirtiendo el conocimiento fáctico en procedimientos que retendré? ¿Estoy sobreaprendiendo los aspectos más críticos de la habilidad?
	
	8. INTUICIÓN. ¿Entiendo profundamente las cosas que estoy aprendiendo, o solo estoy memorizando? ¿Podría enseñar las ideas y los procedimientos que estoy estudiando a otra persona? ¿Tengo claro por qué lo que estoy aprendiendo es cierto, o todo parece arbitrario y sin relación?
	
	9. EXPERIMENTACIÓN. ¿Me estoy atascando con mis recursos y técnicas actuales? ¿Necesito diversificarme y probar nuevos enfoques para alcanzar mi objetivo? ¿Cómo puedo ir más allá de dominar los conceptos básicos y crear un estilo único para resolver problemas de manera creativa y hacer cosas que otros no han explorado antes? 
	
	Juntos, estos principios sirven como direcciones, no como destinos. En cada caso, mire cómo está progresando actualmente a través de sus materiales y vea qué podría hacer de manera diferente. ¿Necesita cambiar los recursos? ¿Necesita ceñirse a los mismos recursos pero dedicar más tiempo a un tipo diferente de práctica? ¿Debería buscar nuevos entornos para la retroalimentación, la franqueza o la inmersión? Todos estos son ajustes sutiles que puede hacer en el camino.
	
	Paso 4: Revisa tus resultados
	
	Una vez que haya terminado su proyecto (o si termina poniéndolo en pausa por algún motivo), debe dedicar un poco de tiempo a analizarlo. ¿Qué salió bien? ¿Qué salió mal? ¿Qué debe hacer la próxima vez para evitar cometer esos mismos errores? No todos sus proyectos tendrán éxito. He tenido proyectos de ultraaprendizaje con los que me sentí bien. He tenido otros que no funcionaron tan bien como esperaba. Aunque la tendencia es culpar a la voluntad y la motivación, muy a menudo los problemas con los proyectos se remontan a su concepción. Trabajé en un proyecto dedicado a mejorar mi coreano, después de mi viaje, invirtiendo cinco horas por semana. No fue tan exitoso como esperaba porque no invertí suficiente tiempo en concentrarme en la práctica directa e inmersiva desde el principio. En cambio, mi método de estudio dependía mucho de los ejercicios de los libros de texto, que eran aburridos y no se transferían demasiado bien al mundo real. Si hubiera pensado un poco más al respecto, habría pasado una semana o dos antes de tiempo tratando de encontrar lugares para practicar, en lugar de tratar de girar a mitad de camino, cuando ya estaba perdiendo algo de motivación. Esta lucha ilustra que dominar los principios es un proceso de toda la vida. Incluso después de muchas experiencias aprendiendo idiomas y sabiendo lo que funciona bien, me deslicé hacia un enfoque menos efectivo porque no planeé mi proyecto adecuadamente. En otros casos, es posible que un proyecto no funcione como esperaba, pero esa lección seguirá siendo valiosa. Empecé con un proyecto para aprender ciencia cognitiva más profundamente, partiendo de una lista de libros. Eventualmente, sin embargo, gran parte de ese proyecto se transformó en un deseo de investigar para este libro, lo que me puso en contacto con mucha ciencia, ahora combinada con una salida para una forma más directa de aplicarla. Incluso sus proyectos exitosos valen la pena analizarlos. A menudo pueden decirle más que sus fallas porque las razones por las que un proyecto exitoso tuvo éxito son los mismos elementos que desea conservar y replicar para el futuro. Con el ultraaprendizaje, como con toda la autoeducación, el objetivo no es simplemente aprender una habilidad o materia, sino perfeccionar y mejorar su proceso de aprendizaje general. Cada proyecto exitoso se puede refinar y mejorar para el próximo.
	
	Paso 5: elija mantener o dominar
	
	Lo que ha aprendido Después de haber aprendido su habilidad y analizado sus esfuerzos, tiene que tomar una decisión. ¿Qué quieres hacer con la habilidad? Sin un plan establecido, la mayor parte del conocimiento eventualmente decae. Esto se puede aliviar un poco siguiendo los principios del ultraaprendizaje. Sin embargo, todo el conocimiento decae sin ningún tipo de intervención, por lo que el mejor momento para tomar una decisión sobre cómo manejará eso es justo después de aprender algo.
	
	Opción 1: Mantenimiento
	
	La primera opción es invertir suficiente práctica para mantener la habilidad pero sin ningún objetivo concreto de llevarla a un nuevo nivel. Esto a menudo se puede lograr estableciendo un hábito de práctica regular, incluso si es mínimo. Como mencioné en el capítulo sobre la retención, una de las preocupaciones que tenía después del año sin proyecto de inglés era que aprender idiomas tan intensamente durante un corto período de tiempo podría conducir no solo a un aprendizaje rápido sino también a un olvido rápido. Como resultado, hice un esfuerzo por continuar practicando después de que terminó el viaje, dedicando treinta minutos a la semana a cada idioma durante el primer año y treinta minutos al mes a cada idioma el año siguiente. Otra opción es tratar de integrar la habilidad en tu vida. Así es como mantengo mis habilidades de programación, donde escribo scripts de Python para manejar tareas de trabajo que de otro modo serían engorrosas o molestas. Este tipo de práctica es más esporádica, pero asegura que la mantendré lo suficiente como para que sea utilizable. Este tipo de uso ligero está lejos de las matemáticas profundas y los algoritmos que aprendí en mis cursos del MIT, pero es suficiente para mantener un pie en la puerta si quiero embarcarme en un proyecto más grande en un momento posterior. El olvido, como lo descubrió Hermann Ebbinghaus hace más de cien años, cae con una curva exponencialmente decreciente. Eso significa que los recuerdos que se retienen por más tiempo tienen cada vez menos probabilidades de ser olvidados cuando hagas un seguimiento en una fecha posterior. Este patrón sugiere que la práctica de mantenimiento también puede disminuir a un ritmo decreciente, por lo que se conservará la mayor parte del conocimiento que ha adquirido. Esto significa que es posible que desee comenzar con un hábito de práctica más seria, pero reducir el tiempo dedicado a ello uno o dos años después de que finalice su proyecto para obtener la mayor parte del beneficio, como hice con los idiomas que estudié.
	
	Opción 2: Reaprendizaje
	
	Olvidar no es ideal, pero para muchas habilidades, los costos de volver a aprender la habilidad más adelante son menores que los costos de mantenerla en forma continuamente. Hay un par de razones para esto. Primero, es posible que haya aprendido más de lo que realmente necesita, por lo que si parte de ese conocimiento decae selectivamente debido al desuso, automáticamente será el conocimiento menos importante que haya adquirido. Estudié muchas materias del MIT que no creo que vuelva a usar, aunque comprender su esencia podría ser útil más adelante. Por lo tanto, mantener actualizada mi capacidad para probar teoremas de lógica modal, por ejemplo, solo tiene un valor marginal. Saber qué es la lógica modal y dónde podría aplicarla en caso de que quiera aprender algo que la requiera probablemente sea suficiente. Reaprender es generalmente más fácil que aprender por primera vez. Aunque el rendimiento en las pruebas cae drásticamente, es probable que el conocimiento sea inaccesible en lugar de olvidado por completo. Esto significa que hacer un curso de actualización o una serie de prácticas puede ser suficiente para reactivar la mayor parte en una fracción del tiempo que tomó aprenderlo inicialmente. Esta puede ser la estrategia óptima para temas que necesita usar con poca frecuencia y para los cuales las situaciones para usarlos no aparecerán sin previo aviso. A menudo, reconocer que un cierto dominio de conocimiento es útil para un tipo de problema en particular es más importante que los detalles para resolver el problema, ya que este último se puede volver a aprender, pero olvidar el primero lo impedirá resolver esos problemas.
	
	Opción 3: Maestría
	
	La tercera opción, por supuesto, es profundizar en la habilidad que ha aprendido. Esto se puede hacer a través de la práctica continua a un ritmo más ligero o siguiendo con otro proyecto de ultraaprendizaje. Un patrón común que he notado en mi propio aprendizaje es que un proyecto inicial cubre un territorio más amplio y algunos conceptos básicos y expone nuevas vías de aprendizaje que antes estaban ocultas. Puede identificar un subtema o rama de habilidad dentro del dominio que estaba aprendiendo antes y que desea seguir. De lo contrario, puede decidir transferir una habilidad aprendida en un lugar a un nuevo dominio. Una de mis metas después de regresar de mi viaje a China era aprender a leer mejor el chino, lo cual había sido solo una meta incidental mientras viajaba allí. El dominio es un largo camino que se extiende mucho más allá de un solo proyecto. A veces, las barreras que superas en tu esfuerzo inicial son suficientes para despejar el camino a un lento proceso de acumulación para llegar al dominio final. En muchos dominios, comenzar es bastante frustrante, por lo que es difícil practicar sin una cierta cantidad de esfuerzo. Sin embargo, una vez que se alcanza ese umbral, el proceso pasa a ser uno de acumulación de grandes franjas de conocimiento y, por lo tanto, puede avanzar a un ritmo más paciente. Por otro lado, algunos proyectos se atascarán y tendrás que dedicar tiempo a desaprender y superar tus frustraciones nuevamente para salir adelante. Ese tipo de proyectos se benefician más de los métodos precisos y agresivos del ultraaprendizaje para alcanzar el dominio final.
\end{verbatim}

