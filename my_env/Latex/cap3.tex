\part{Matem\'aticas}

\chapter{Ecuaciones diferenciales}
En est\'e cap\'itulo vamos a entender que es una ecuaci\'on diferencial. 

Para comenzar a formular lo que es una ecuaci\'on diferencial, comencemos explorando lo que es la en s\'i, la palabra ecuaci\'on, de donde proviene, ?`C\'omo hemos trascendido los seres humanos en el pensamiento de las ecuaciones? ?`Para que sirven las ecuaciones y por ende donde aplicarlas? Es un tarea muy dura, pero lo vamos a lograr, formulando  preguntas para ir llegando a respuestas y de estas respuestas dar el salto a lo que yo llamo familiarizaci\'on intelectual. 

?`Por qu\'e planteamos ecuaciones? tratemos de acercarnos a esta pregunta observando la siguiente ecuaci\'on
\begin{equation}\label{equ1.1}
	y+5 = 0
\end{equation}

?`Qu\'e elementos involucrados conocemos en la ecuaci\'on \ref{equ1.1}?

Los n\'umeros 5 y 0, son dos enteros y la variable $y$ representando lo desconocido en la ecuaci\'on, es decir, lo que queremos encontrar, si nos enfocamos en ese sentido, llegaremos a que $y=-5$. 
\begin{equation}
	y(x)=y^{\prime}(x)
\end{equation}
Note que tenemos una igualdad de funciones, una ecuaci\'on que se cumple solo para las funciones que son soluciones de la misma. ?`Cuales son estas funciones? las funciones $c\exp(x)$, de esta manera podemos definir este tipo de ecuaciones, ?`Por qu\'e es importante las ecuaciones diferenciales? ?`Qu\'e fen\'omenos describe el c\'alculo? en intervengan, la distancia, velocidades y aceleraciones.

\begin{definicion}[Ecuaci\'on diferencial]
Una ecuaci\'on diferencial es cualquier ecuaci\'on que contiene las derivadas de una o m\'as variables dependientes con respecto a una o m\'as variables independientes.
\end{definicion}
\begin{ejemplo}
	\begin{align}
		\frac{dy}{dx} + \frac{dz}{dx} &= 2y + z, \\
		\frac{\partial^{2}u}{\partial{x^{2}}} &= \frac{\partial^{2}u}{\partial t^{2}} - 2\frac{\partial u}{\partial t} 
	\end{align}
\end{ejemplo}


\part{Ingles}

\chapter{Frases en ingl\'es}

\begin{description}
	\item[After you: ] Despu\'es de usted.
	\item[Check it out: ] \'Echale un vistazo.
	\item[I guess so: ] Supongo que si.
	\item[What a mess: ] Qu\'e desastre.
	\item[Good for you: ] Bien por ti.			
	
	to get one is hopes up 
\end{description}

\chapter{ Unknown words }
\begin{description}
	\item[Which: ]  cual ; pronombre: que.
	\item[Shall: ]  debera.
	\item [use:  ]  usar.
	\item[brief: ]  breve.
	\item[either:]	conjuci\'on : o, adjetive: Cualquiera de los dos, uno u otro; Adverbio: tambi\'en.
	\item[Roughly speaking: ] mas o menos
	\item[behave: ] comportarse
	\item[obey: ] obedecer
	\item[listed: ] listado
	\item[in that: ] en eso
	\item[for most:] para la mayor\'ia
	\item[may :] poder, ser posible.
	\item[well :] bien
	\item[ to allow for this :] para permitir esto
	\item[rather than  :] m\'as bien que
	\item[itself :] s\'i mismo
	\item[this means :] esto significa
	\item[so are :] tambi\'en lo son
	\item[such :] tal
	\item[below :] abajo
	\item[performance :] actuaci\'on, el rendimiento, desempe\~no
	\item[us out of :]	nosotros fuera de							
	\item[given :]	dado								
	\item[on this	 :]	en este							
	\item[unless :] a no ser que			
	\item[must  :] deber
	\item[we leave it :] lo dejamos
	\item[want  :] desear
	\item[dwell :] residir
	\item[happen :] suceder
	\item[least :] el menos
	\item[Often :] Con frecuencia
	\item[exposure :] exposici\'on
	\item[better :] mejor
	\item[worry :] preocuparse
	\item[about :] sobre						
	\item[vowels :] 
	\item[consonants :]

\end{description}

\chapter{Ingl\'es A1}

Marco Com\'un Europeo de Referencia para las Lenguas (MCER): 

\section{An\'alisis de habilidades}

A continuaci\'on, encontramos un an\'alisis de los diferentes componentes del idioma y las habilidades individuales que un estudiante deber\'ia tener para recibir un certificado A1 de ingles: 

{\bf Compresi\'on auditiva}

Los estudiantes de nivel A1 deberi\'an ser capaces de entender un ingl\'es est\'andar simple, siempre y cuando sea hablado con claridad por alguien paciente y dispuesto ayudar. Especificamente, deberi\'an poder reconocer frases y palabras comunes, relacionadas con ellos mismos, su entorno y aquellos cercanos a ellos. Adem\'as, deber\'ian entender cosas como los n\'umeros, direcciones y otras instrucciones muy b\'asicas en ingl\'es.

Welcome everyone to the basic English course. A one for beginners. I'm your teacher, kira Sage, nice to meet you. Who am i? I am from the U.S.A I have taught C level executives and software engineers at companies like google skhynix, Tesla, Sony and more. I have nine years of teaching experience in countries like Usa and Japan. Here are a couple of interesting facts about me. I have visited four of the six Disneyland's around the world. This is me at Star Wars, Galaxy edge in Anaheim California. Also i have Master's degree in teaching T. Saw that means I can teach teachers and learners like you in this course you will learn the alphabet ah and un sentences with it's plural forms, sentences with there are numbers, colors, subject pronouns, professions, greetings, negative and interrogative statements, possessive adjectives. Days of the week, your hobbies, questions with what's your and to review everything from this course we'll have to wrap up class. Here's you'll learn all of the course concepts through worksheets, find the worsheets in the resources section through interactive quizzes and through interactive explanations on an imaginary website called plattso plattso is where we practice english so we can make friends online. Are you ready for this  fun adventure? I'll see you in the next class.


Bienvenido todos al curso the Ingles b\'asico. A1 para comenzar 



\begin{verbatim}
Intento 1: 
Welcome everyone to the Black Sea English course A1 for beginners I
 am here t-shirt carousel nice to me you Hoang I who on my I'm from
 the USA I have to see level executive a software engineer and companies
 like Google sky and Tesla Sony and more I cannae jeers of teaching experience in countries like USA and Japan he are Kobe of interesting
 facts about me I had visits of the seeds Disneyland around the world this is me at Star Wars Galaxy Edge in Anaheim California also I have
 massive degree in teaching t so that means can teach t-shirts in Learners like you in this course you will learn the alphabet I'm on
 sentence with it's pure Air Force sentence with there are numbers caller surgeon pronounce profession where teens interrogative
 status. Osiria Jetties days of the weave your hobbies question with what you ain't to review everything from this course we wait
 we all have to grab a glass of the course Concepts world war ships find a wall sheets in the results section throw in charity
 crisis and throw into righteous play Nations oh imagine a website called black English so we can make friends only are you
 ready for this funeral bencher Adventure I also see you in the next class.
         
         
 Segundo intento: 
 welcome everyone to the basic English course. a wife for beginners at
  Food dock. Avon for beginners sorry please thought a blunt for 
  beginners. A1 for beginners. I am your teacher call Ma Kira say Asahi 
  call Mama, nice to meet you. who I who am I. I am from the USA. I 
  hopped out I have towels. IHOP takeout see Liberty Security in Sulphur 
  and genius and companies like Google Skynet call mistakes Le gamma 
  Sony a moor Anne Moore Anne Moore Sony and more. I have 9 years of 
  teaching experience in countries like USA in Japan. Cher here. T R A 
  Copley. here are a couple off interesting facts about me. I had visit 
  visit. I had visited full of the sea Disneyland around the world. this 
  is me and it's sad worse, Galaxy Edge in Anaheim California. also I 
  have. also I had Mercer metzer Meister Meister mustard master. also I 
  heart muscle is degree and teach t. all soup IHOP Massillon he's they 
  greet and teach tea. soda means I can teach t-shirts in layers
\end{verbatim} 

\url{https://www.youtube.com/watch?v=EVNhOAEW784}

\url{https://howjsay.com/how-to-pronounce-taught}
Para pronuciar Taught comience colocando su lengua detr\'as de sus dientes superiores y agregue el sonidoo AH corto y abierto termine con una T colocando su lengua detr\'as de sus dientes superiores



Numbers class Two: 

Welcome to the 1st Module. This english fundamentals in this module.
In this Module you will learn the 
\begin{enumerate}
	\item alphabet.
	\item A and N 
	\item sentences with it's plural forms
	\item  and sentences with there are
\end{enumerate}
 this is the alphabet. Now let's go and learn on platzo.
 
 Welcome to plattzo. Now loading oh do you know what this is?
 
 it's capture. We need letters to unlock the capture. Well let's learn in this class. 
 
 You will learn the letters of the english alphabet. In the English Alphabet there are 26 letters. Let's practice. I will say a letter an repeat after me. Ready, let's go. A B Bravo. C D E. Excellent F G. Googd job H I J K, keep it up. L M magnificent N. Nice work. O P perfect Q R S super T terrific. U V very goog W wonderful X. Why Z in some countries this letter is zed.
 
 Now, you know the letters of the alphabet? A B C D E F G H I J K L M N O P Q R S T u v w x, y z. OR ZED. 	Let's use these letters to unlock the capture on plattzo.
 
 The capture says a phrase, not a robot. Let's spell the phrase letter by letter N O T A R O B O T. You did it. You unlock the capture. 
 
 Great job. it's your turn. Here is the alphabet but some letters are missing. 
 
 Type, the missing letters in the discussion panel. 
 Don't forget to download your worksheet from the resources section that worksheet is your plattzo profile. Complete the user name area of your plattzo profile on your woorksheet, keep this worksheet with you at all times. We will complete your plattzo profile in each module of this course. So keep ir ready for extra practice. Type your user name in the discussion panel.
 
 My user name is Kira K I R A. I'll see you in the next class.
 
 
 Email: jcausilmartinez@gmail.com
 
 Number phone: 314 796 81 19.
 
 User name: Javier Andres Causil Martínez


My introducction in is Spanish:  

Hola mi nombre es Javier Andr\'es Causil Mart\'inez, soy de Colombia del departamento de Antioquia en la ciudad de Caucasia. Tengo 27 a\~nos, soy Matem\'atico  de la Universidad de Antioquia y programador, me gusta la m\'usica, hacer deporte, aprender cosas nuevas. Siento gran pasi\'on por la m\'usica cl\'asica. 



\subsection{A and AN}

A and AN refer to only one person, place, or thing, or noun.

An is an article that we use with vowels. Vowels are ``a'', ``e'', ``i'', ``o'', ``u''

Use ``an'' with these vowels. Now, let's them together on Platzo.

``A'' is the article we use with consonants.

Be careful! there is an exception. The most common exception is with the letter U.

Now you can use ``a'' and ``an'' with vowels and consonants.

\subsection{IT'S SENTENCES}

In this class, you will learn sentences with ``it's''.

First, use ``it's'' or ``'it is' with a noun, 

it is + a person: a thing: an item: a place

We can make sentences with ``it's, an article, and a noun''.

In this class, you wil learn forms plurals, 

\subsection{Plural forms}

The most common plural form is to add ``s'' at the end of the word. Para otras palabras que terminan con ``S,SH,CH,X'' y ``Z'', we add ``es'' to the end of those words.

Words with ``y'' are special. because when a ``y'' is next to a vowel, we just add ``S'', when a ``y'' is next to a consonant, we change the ``y'' to ``i'' and add ``es''.

\part{Lecturas de libros}

\section{Ultralearning}

Comienza con la ilusi\'on de estudiar en el MIT, pero t\'ermina estudiando negocios en la Universidad de Manitoba, una escuela Canadiense de rango medio, ya que era la que pod\'ia pagar. Pero al terminar la carrera concluyo que se equivoco de carrera cuatro a\~nos mas tarde se da cuenta de una especializaci\'on en administraci\'on y de inform\'atica 

\subsection{Why Ultralearning Matters}

Cu\'ando aplicar el ultraaprendizaje 

\begin{enumerate}
	\item es seguir ultralearning a tiempo parcial.
	\item es buscar el ultraaprendizaje durante las brechas en el trabajo y la escuela.
	\item Es integrar los principios del ultraaprendizaje en el tiempo y la energía que ya dedica al aprendizaje.
	
\end{enumerate}


\subsection{El valor del ultraaprendizaje}

La capacidad de adquirir habilidades duras de manera efectiva y eficiente es inmensamente valiosa. No solo eso, sino que las tendencias actuales en econom\'ia, educaci\'on y tecnolog\'ia van a exacerbar la diferencia entre quienes tienen esta habilidad y quienes no la tienen.

\subsection{Como convertirse en un ultraaprendiz}

Primeros pasos de un ultraaprendiz novato


Principios para convertirse en ultraaprendiz


\begin{verbatim}
1. METALEARNING: PRIMERO DIBUJAR UN MAPA. Comience por aprender cómo aprender el tema o la habilidad que desea abordar. Descubra cómo hacer una buena investigación y cómo aprovechar sus competencias pasadas para aprender nuevas habilidades más fácilmente.

2. ENFOQUE: AFILA TU CUCHILLO. Cultivar la capacidad de concentración. Dedique períodos de tiempo en los que pueda concentrarse en el aprendizaje y haga que sea fácil hacerlo.

3. SERIEDAD: VAYA DERECHO. Aprende haciendo aquello en lo que quieres ser bueno. No lo cambie por otras tareas, solo porque son más convenientes o cómodas.

4. EJERCICIO: ATACA TU PUNTO MÁS DÉBIL. Sea implacable en la mejora de sus puntos más débiles. Divida las habilidades complejas en partes pequeñas; luego domine esas partes y vuelva a construirlas juntas.

5. RECUPERACIÓN: PRUEBA PARA APRENDER. Las pruebas no son simplemente una forma de evaluar el conocimiento, sino una forma de crearlo. Ponte a prueba antes de sentirte seguro y oblígate a recordar activamente la información en lugar de revisarla pasivamente.

6.COMENTARIOS: NO EVITE LOS GOLPES. La retroalimentación es dura e incómoda. Sepa cómo usarlo sin dejar que su ego se interponga en el camino. Extraiga la señal del ruido, para que sepa a qué prestar atención y qué ignorar.

7. RETENCIÓN: NO LLENE UN CUBO CON FUGAS. Entiende lo que olvidas y por qué. Aprende a recordar las cosas no solo por ahora sino para siempre.

8. INTUICIÓN: PROFUNDICE ANTES DE CONSTRUIR. Desarrolla tu intuición a través del juego y la exploración de conceptos y habilidades. Entiende cómo funciona el entendimiento, y no recurras a trucos baratos de memorización para evitar saber las cosas en profundidad.

9. EXPERIMENTACIÓN: EXPLORA FUERA DE TU ZONA DE CONFORT. Todos estos principios son solo puntos de partida. El verdadero dominio proviene no solo de seguir el camino recorrido por otros, sino de explorar posibilidades que aún no han imaginado.
\end{verbatim}

\section{Principio 1}

\subsection{What Is Metalearning?}

Aprehender sobre el aprendizaje

El poder de su mapa de metaaprendizaje

Ser capaz de ver c\'omo funciona un tema, qu\'e tipo de habilidades e informaci\'on se deben dominar y qu\'e m\'etodos est\'an disponibles para hacerlo de manera m\'as efectiva es la base del \'exito de todos los proyectos de ultraaprendizaje.

\begin{verbatim}
Una vez que haya entendido por qué está aprendiendo, puede comenzar a ver cómo se estructura el conocimiento en su materia. Una buena manera de hacer esto es escribir en una hoja de papel tres columnas con los títulos “Conceptos”, “Hechos” y “Procedimientos”. Luego haga una lluvia de ideas sobre todas las cosas que necesitará aprender. No importa si la lista está perfectamente completa o precisa en esta etapa. Siempre puedes revisarlo más tarde. Su objetivo aquí es obtener un primer pase difícil. Una vez que comience a aprender, puede ajustar la lista si descubre que sus categorías no son del todo correctas.
\end{verbatim}

Primera columna de conceptos  si algo necesita ser entendido, no solo memorizado, lo pongo en esta columna en lugar de la segunda columna de hechos.
\begin{verbatim}
	1. METALEARNING: PRIMERO DIBUJAR UN MAPA. Comience por aprender cómo aprender el tema o la habilidad que desea abordar. Descubra cómo hacer una buena investigación y cómo aprovechar sus competencias pasadas para aprender nuevas habilidades más fácilmente.
	
	2. ENFOQUE: AFILA TU CUCHILLO. Cultivar la capacidad de concentración. Dedique períodos de tiempo en los que pueda concentrarse en el aprendizaje y haga que sea fácil hacerlo. 
	
	3. SERIEDAD: VAYA DERECHO. Aprende haciendo aquello en lo que quieres ser bueno. No lo cambie por otras tareas, solo porque son más convenientes o cómodas.
	
	4. EJERCICIO: ATACA TU PUNTO MÁS DÉBIL. Sea implacable en la mejora de sus puntos más débiles. Divida las habilidades complejas en partes pequeñas; luego domine esas partes y vuelva a construirlas juntas.
	
	5. RECUPERACIÓN: PRUEBA PARA APRENDER. Las pruebas no son simplemente una forma de evaluar el conocimiento, sino una forma de crearlo. Ponte a prueba antes de sentirte seguro y oblígate a recordar activamente la información en lugar de revisarla pasivamente. 
	
	6.COMENTARIOS: NO EVITE LOS GOLPES. La retroalimentación es dura e incómoda. Sepa cómo usarlo sin dejar que su ego se interponga en el camino. Extraiga la señal del ruido, para que sepa a qué prestar atención y qué ignorar.
	
	7. RETENCIÓN: NO LLENE UN CUBO CON FUGAS. Entiende lo que olvidas y por qué. Aprende a recordar las cosas no solo por ahora sino para siempre. 
	
	8. INTUICIÓN: PROFUNDICE ANTES DE CONSTRUIR. Desarrolla tu intuición a través del juego y la exploración de conceptos y habilidades. Entiende cómo funciona el entendimiento, y no recurras a trucos baratos de memorización para evitar saber las cosas en profundidad. 
	
	9. EXPERIMENTACIÓN: EXPLORA FUERA DE TU ZONA DE CONFORT. Todos estos principios son solo puntos de partida. El verdadero dominio proviene no solo de seguir el camino recorrido por otros, sino de explorar posibilidades que aún no han imaginado. 
\end{verbatim}

\section{Principio 1}

\subsection{What Is Metalearning?}

Aprehender sobre el aprendizaje

El poder de su mapa de metaaprendizaje

Ser capaz de ver c\'omo funciona un tema, qu\'e tipo de habilidades e informaci\'on se deben dominar y qu\'e m\'etodos est\'an disponibles para hacerlo de manera m\'as efectiva es la base del \'exito de todos los proyectos de ultraaprendizaje.

\begin{verbatim}
	Una vez que haya entendido por qué está aprendiendo, puede comenzar a ver cómo se estructura el conocimiento en su materia. Una buena manera de hacer esto es escribir en una hoja de papel tres columnas con los títulos “Conceptos”, “Hechos” y “Procedimientos”. Luego haga una lluvia de ideas sobre todas las cosas que necesitará aprender. No importa si la lista está perfectamente completa o precisa en esta etapa. Siempre puedes revisarlo más tarde. Su objetivo aquí es obtener un primer pase difícil. Una vez que comience a aprender, puede ajustar la lista si descubre que sus categorías no son del todo correctas.
	
	Primera columna de conceptos  si algo necesita ser entendido, no solo memorizado, lo pongo en esta columna en lugar de la segunda columna de hechos.
	
	En la segunda columna, escribe todo lo que necesites memorizar. Los hechos son cualquier cosa que sea suficiente si puedes recordarlos. No es necesario que los entienda demasiado profundamente, siempre y cuando pueda recordarlos en las situaciones adecuadas. 
	
	En la tercera columna, escribe todo lo que necesites practicar. Los procedimientos son acciones que deben realizarse y es posible que no impliquen mucho pensamiento consciente.
	
	Una vez que haya terminado su lluvia de ideas, subraye los conceptos, hechos y procedimientos que serán más desafiantes. Esto le dará una buena idea de cuáles serán los principales cuellos de botella de aprendizaje y puede comenzar a buscar métodos y recursos para superar esas dificultades.
	
	Ahora que ha respondido dos preguntas, por qué está aprendiendo y qué está aprendiendo, es hora de responder la pregunta final: ¿Cómo va a aprenderlo? Sugiero seguir dos métodos para responder cómo aprenderá algo: Benchmarking y el Método de énfasis/exclusión. (Benchmarking and the Emphasize/Exclude Method.)
	
	Las luchas con el enfoque que tienen las personas generalmente vienen en tres variedades amplias: comenzar, mantener y optimizar la calidad del enfoque de uno. Los ultraalumnos son implacables a la hora de encontrar soluciones para manejar estos tres problemas, que forman la base de la capacidad de concentrarse bien y aprender profundamente.
	
	 Examinemos algunas de las tácticas que usan los ultraestudiantes para maximizar este principio y aprovechar las deficiencias de la educación más típica. 
	 
	 Táctica 1: aprendizaje basado en proyectos Muchos ultraestudiantes optan por proyectos en lugar de clases para aprender las habilidades que necesitan.
	 
	 Táctica 2: Aprendizaje inmersivo
	 
	 La inmersión es el proceso de rodearse del entorno objetivo en el que se practica la habilidad. Esto tiene la ventaja de requerir una cantidad de práctica mucho mayor de lo que sería típico, además de exponerlo a una gama más completa de situaciones en las que se aplica la habilidad.
	 
	 Táctica 3: El método del simulador de vuelo
	 
	 Es simular el entorno, aproximarse al entorno real a través de uno muy parecido.
	 
	 Táctica 4: El enfoque exagerado
	 
	 El último método que encontré para mejorar la franqueza es aumentar el desafío, de modo que el nivel de habilidad requerido esté completamente contenido dentro de la meta establecida. 
	 
	 Aprenda directamente de la fuente
	 
	 El aprendizaje directo es uno de los sellos distintivos de muchos de los proyectos exitosos de ultraaprendizaje que he encontrado, particularmente por lo diferente que puede ser del estilo de educación al que la mayoría de nosotros estamos acostumbrados.
	 
	 Los ultraestudiantes, por el contrario, emplean con frecuencia lo que llamaré el enfoque Direct-Then-Drill.
	 
	 El primer paso es tratar de practicar la habilidad directamente. Esto significa averiguar dónde y cómo se usará la habilidad y luego tratar de igualar esa situación lo más cerca posible al practicar. 
	 
	  El siguiente paso es analizar la habilidad directa y tratar de aislar los componentes que son pasos determinantes en su rendimiento o subhabilidades que encuentra difíciles de mejorar porque hay demasiadas otras cosas sucediendo para que usted se concentre en ellas.
	  
	  El paso final es volver a la práctica directa e integrar lo que has aprendido. Esto tiene dos propósitos. La primera es que, incluso en ejercicios bien diseñados, habrá problemas de transferencia debido al hecho de que lo que antes era una habilidad aislada debe trasladarse a un contexto nuevo y más complejo. Piense en esto como si estuviera construyendo el tejido conectivo para unir los músculos que fortaleció por separado. La segunda función de este paso es verificar si su ejercicio fue bien diseñado y apropiado. Muchos intentos de aislar un ejercicio pueden terminar en fracaso porque el ejercicio realmente no llega al corazón de lo que era difícil en la práctica real. 
	  
	  Tácticas para diñar simulacros: 
	  
	  
	 
\end{verbatim}


