\part{Mi pc}

La inspiraci\'on no se puede provocar pero se puede estimular realizando tareas simples, antes de cometer otras mas complejas en cualquier caso es un fen\'omeno instintivamente humano y distinto para cada persona.
\chapter{Instalando programas en mi pc}
\section{Terminal}
Para instalar node ejecuto el comando 
\begin{verbatim}
	sudo apt-get install nodejs
\end{verbatim}
Este node es el ultimo estable que se encuentra en los repositorios de Ubuntu, pero descargar la ultima versi\'on ejecutando los siguientes comandos:

\begin{verbatim}
	javier@javier-Lenovo-G40-80:~$ curl -o- https://raw.githubusercontent.com/nvm-sh/nvm/v0.35.3/install.sh | bash
	
	
	javier@javier-Lenovo-G40-80:~$ source ~/.bashrc
	
	javier@javier-Lenovo-G40-80:~$ nvm list-remote
	
	javier@javier-Lenovo-G40-80:~$ nvm install v16.15.0
	
	javier@javier-Lenovo-G40-80:~$ sudo apt-get install npm
	
	
\end{verbatim}

La versi\'on es la necesaria. esa la encontramos en la pagina de nodejs.

Figma: 
\begin{verbatim}
	javier@javier-Lenovo-G40-80:~$ sudo snap install figma-linux
\end{verbatim}

git: 
\begin{verbatim}
	sudo add-apt-repository ppa:git-core/ppa 
	
	sudo apt update 
	
	sudo apt install git
\end{verbatim}
\begin{verbatim}
	javier@javier-Lenovo-G40-80:~$ sudo apt-get update
	Obj:1 http://co.archive.ubuntu.com/ubuntu jammy InRelease
	Obj:2 http://co.archive.ubuntu.com/ubuntu jammy-updates InRelease
	Obj:3 http://security.ubuntu.com/ubuntu jammy-security InRelease
	Obj:4 http://co.archive.ubuntu.com/ubuntu jammy-backports InRelease
	Leyendo lista de paquetes... Hecho
	javier@javier-Lenovo-G40-80:~$ python3
	
	javier@javier-Lenovo-G40-80:~$ sudo apt-get update
	Obj:1 http://co.archive.ubuntu.com/ubuntu jammy InRelease
	Des:2 http://security.ubuntu.com/ubuntu jammy-security InRelease [110 kB]
	Des:3 http://co.archive.ubuntu.com/ubuntu jammy-updates InRelease [109 kB]
	Obj:4 http://co.archive.ubuntu.com/ubuntu jammy-backports InRelease	
\end{verbatim}

Intalando MySQl 

\begin{verbatim}
	https://dev.to/gsudarshan/how-to-install-mysql-and-workbench-on-ubuntu-20-04-localhost-5828 
	
	https://dev.mysql.com/downloads/file/?id=509020 
	
	https://ubuntu.com/server/docs/databases-mysql
\end{verbatim}


\section{Programas instalados desde software Ubuntu}
Texstudio, vscode, Mysql Workbench Community

\part{Python}

Ejecutar Jupyter dentro del entorno virtual, 

\url{https://es.acervolima.com/uso-de-jupyter-notebook-en-un-entorno-virtual/}

Una vez creado el entorno virtual ejecutamos 

\begin{verbatim}
	https://es.acervolima.com/uso-de-jupyter-notebook-en-un-entorno-virtual/
\end{verbatim}
abrimos el Jupyter Notebook y procedemos a cambiar el kernes en la opcion venv que es como se llamo nuestro kernel, de esta manera podemos acceder a todos los paquetes instalados en nuestro entorno virtual. Para desactivarlo procedemos a ejecutar el siguiente comando: 
\begin{verbatim}
	jupyter-kernelspec uninstall venv
\end{verbatim}

En si solo es cambiar la direcci\'on del kernel.
\chapter{Numpy}

\section{Dimensiones en matrices}

\subsection{0-D Arrays}
0-D arrays, or scalars, are the elements in a array. Each value in an array is a 0-D array.

\begin{verbatim}
	import numpy as np
	
	arr = np.array(42)
	
	print(arr) 
	
	ouput()
			42
\end{verbatim}


\subsection{1-D Arrays}
An array that has 0-D arrays as its elements is called uni-dimensional or 1-D array.

\begin{verbatim}
Example

Create a 1-D array containing the values 1,2,3,4,5:
import numpy as np

arr = np.array([1, 2, 3, 4, 5])

print(arr) 
print(type(arr)) 
output()
   [1 2 3 4 5]
   <class 'numpy.ndarray'>
\end{verbatim}
No confundir con una lista de Python son totalmente diferentes.


\subsection{2-D Arrays}
An array that has 1-D arrays as its elements is called a 2-D array.

These are often used to represent matrix or 2nd order tensors.

NumPy has a whole sub module dedicated towards matrix operations called numpy.mat

\begin{verbatim}
Example

Create a 2-D array containing two arrays with the values 1,2,3 and 4,5,6:
import numpy as np

arr = np.array([[1, 2, 3], [4, 5, 6]])

print(arr) 
\end{verbatim}




\subsection{3-D Arrays}
An array that has 2-D arrays (matrices) as its elements is called 3-D array.

These are often used to represent a 3rd order tensor.

\begin{verbatim}
Example

Create a 3-D array with two 2-D arrays, both containing two arrays with the values 1,2,3 and 4,5,6:
import numpy as np

arr = np.array([[[1, 2, 3], [4, 5, 6]], [[1, 2, 3], [4, 5, 6]]])

print(arr) 

output()
[ [
  [1 2 3]
  [4 5 6]
  ]
  [
  [1 2 3]
  [4 5 6]
  ]
]
\end{verbatim}


Check Number of Dimensions?

NumPy Arrays provides the ndim attribute that returns an integer that tells us how many dimensions the array have.


\begin{verbatim}
	Example
	
	Check how many dimensions the arrays have:
	import numpy as np
	
	a = np.array(42)
	b = np.array([1, 2, 3, 4, 5])
	c = np.array([[1, 2, 3], [4, 5, 6]])
	d = np.array([[[1, 2, 3], [4, 5, 6]], [[1, 2, 3], [4, 5, 6]]])
	
	print(a.ndim)
	print(b.ndim)
	print(c.ndim)
	print(d.ndim) 
	output()
	0
	1
	2
	3
\end{verbatim}

Higher Dimensional Arrays

An array can have any number of dimensions.

When the array is created, you can define the number of dimensions by using the ndmin argument.

\begin{verbatim}
	Example
	
	Create an array with 5 dimensions and verify that it has 5 dimensions:
	import numpy as np
	
	arr = np.array([1, 2, 3, 4], ndmin=5)
	
	print(arr)
	print('number of dimensions :', arr.ndim) 
	
	output()
	[[[[[1 2 3 4]]]]]
	number of dimensions : 5
\end{verbatim}
In this array the innermost dimension (5th dim) has 4 elements, the 4th dim has 1 element that is the vector, the 3rd dim has 1 element that is the matrix with the vector, the 2nd dim has 1 element that is 3D array and 1st dim has 1 element that is a 4D array.

\section{Numpy data types}

Data Types in Python

By default Python have these data types:

strings - used to represent text data, the text is given under quote marks. e.g. ``ABCD''


integer - used to represent integer numbers. e.g. -1, -2, -3

float - used to represent real numbers. e.g. 1.2, 42.42

boolean - used to represent True or False.

complex - used to represent complex numbers. e.g. 1.0 + 2.0j, 1.5 + 2.5j

Data Types in NumPy

NumPy has some extra data types, and refer to data types with one character, like i for integers, u for unsigned integers etc.

Below is a list of all data types in NumPy and the characters used to represent them.

i - integer\\
b - boolean\\
u - unsigned integer\\
f - float\\
c - complex float\\
m - timedelta\\
M - datetime\\
O - object\\
S - string\\
U - unicode string\\
V - fixed chunk of memory for other type ( void )\\


Shape of an Array

The shape of an array is the number of elements in each dimension.

Get the Shape of an Array

NumPy arrays have an attribute called shape that returns a tuple with each index having the number of corresponding elements.

The shape arrays is return tupla where, first element is 

L aforma de un arreglo 


Estudiando de 
\url{https://www.w3schools.com/python/numpy/numpy_array_join.asp} y python para Data Science
\part{Lenguajes del frontend}

\chapter{Figma}

\chapter{JavaScript}


Funciones de Flecha, 

ES5 //JSON.parse(JSON.stringify(paddockType));

\chapter{Typescript}

\chapter{Next.js}

El objetivo de aprender este lenguaje es darle asistencia a la p\'agina de As-research, junto con el blog. 

Me estare guiando de \url{https://nextjs.org/docs/getting-started}

Foro de discusiones: \url{https://github.com/vercel/next.js/discussions}

Requerimientos del sistema: 

Node.js 12.22.0 or later\\
MacOS, Windows (including WSL), and Linux are supported


C\'omo iniciar un proyecto?

Para construir una aplicaci\'on web completa con React desde cero, hay muchos detalles importantes que tu necesitas considerar: 
\begin{enumerate}
	\item C\'odigo tiene que ser empaquetado usando un empaquetador como webpack y transformarlo utilizando un copilador como Babel.
	\item Debes realizar optimizaciones de producci\'on, como la divisi\'on de c\'odigo.
	\item Es posible que desee renderizar previamente est\'aticamente algunas p\'aginas para mejorar el rendimiento y el SEO. Tambi\'en es posible que desee utilizar la representaci\'on del lado del servidor o la representaci\'on del lado del cliente.
	
	
	\item Es posible que deba escribir alg\'un c\'odigo del lado del servidor para conectar su aplicaci\'on React a su almac\'en de datos.
\end{enumerate}
\begin{verbatim}
npx create-next-app nextjs-blog

Para recorrer el servidor del proyecto es:

npm run dev
\end{verbatim}





